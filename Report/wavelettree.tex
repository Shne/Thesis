\section{Wavelet Tree}

[What is a wavelet tree; theory; usage; rank/select; etc...]

\subsection{What is a Wavelet Tree}
The Wavelet Tree is a binary balanced tree structure, that was invented by Grossi, Grupta and Vitter (TODO: reference) in 2003. It has applications in many areas; from string processing to geometry, and can for instance be used to represent; a sequence of elements, a reordering of elements and a grid of points. 

The Wavelet Tree is actually a generalization of parts of an old data structure by Chazelle (TODO: reference), that is used a lot in Computational Geometry. When Grossi et al. invented the Wavelet Tree, it was a milestone in compressed full-text indexing even though it is mentioned very little in the paper.

\subsection{Theory: Data Structure}
A Wavelet Tree stores a string by creating a bitmap that describes the string using the alphabet of the string. The alphabet is split in the middle and the symbols to the left gets bit value 0 and the symbols to the right gets bit value 1 so that there is a bit for each symbol in the string in the bitmap. The symbols of the string that has bit value 0 is concatenated in the order they have in the string and is added to the left sub-tree and the ones with bit value 1 is added to the right sub-tree. 

This process continues in each sub tree until we end up in the leaves where the string only consists of one unique symbol from the alphabet. An example of a Wavelet Tree can be seen in Figure \ref{fig:WaveletTreeExample}.

\begin{mdframed}[nobreak]
\textbf{Definition:} String representation in a Wavelet Tree

Let $S[1,n] = S_1 S_2 ... S_n$ be a sequence of symbols where $s_i \in \Sigma$ and $\Sigma = [1 .. \sigma]$ is the alphabet. $S$ can then be represented in plain form using $n \lceil \log \sigma \rceil = n \log \sigma = O(n)$ bits.
\end{mdframed}


\begin{figure}[ht!]
\caption{Wavelet Tree on string \textit{adsfadaadsfaads}}				
\Tree
%root
[.adsfadaadsfaads\\001100000110001 !\qsetw{5cm} 
	%left child
	[.adadaadaad\\0101001001 !\qsetw{5cm}
		%left -> left,right child 
		[.aaaaaa\\000000 !\qsetw{5cm} ] [.dddd\\1111 !\qsetw{5cm} ]] 
	%right child
	[.sfsfs\\10101 !\qsetw{5cm} 
		%right -> left,right child
		[.ff\\00 !\qsetw{5.3cm} ] [.sss\\111 !\qsetw{5.3cm} ]]] 
\vspace{1 cm}
\label{fig:WaveletTreeExample}
\end{figure}

		
A Wavelet Tree can be described recursively over a sub-alphabet range $[a .. b] \subseteq [1 .. 0]$, for a sequence $S[1,n]$ over alphabet $[1 .. \sigma]$. A Wavelet Tree over alphabet $[a .. b]$ is a binary balanced tree with $b - a + 1$ leaves. If $a = b$ then the Wavelet Tree is simply a leaf labelled a. Otherwise it has an internal root node $v_{root}$ that represents the string $S[1,n]$. $v_{root}$ stores bitmap $B_{v_{root}}$ in the following way:
\vspace{0.5 cm}

\noindent\rule{\textwidth}{0.5pt}
\begin{algorithmic}
\Function{BitmapConstruction}{$S$}
\If{ $S[i] \leq (a + b)/2$ }
	\State $B_{v_{root}}[i] \gets 0$
\Else
	\State $B_{v_{root}}[i] \gets 1$
\EndIf
\EndFunction
\end{algorithmic}
\noindent\rule{\textwidth}{0.5pt}
\linebreak

We now know how to construct the bitmap in the root of the Wavelet Tree. Now we define how symbols of the string is split into the right- and left sub-tree.\\

\begin{mdframed}[nobreak]
\textbf{Definition:} Splitting string into right and left sub-tree and sub-trees ar also Wavelet Trees. \linebreak

\noindent
Let $S_0[1,n_0] =$ subsequence of $S[1,n]$ formed by symbols $c \leq (a + b)$/2.

\noindent
Let $S_1[1,n_1] =$ subsequence of $S[1,n]$ formed by symbols $c > (a + b)$/2.
\\ \linebreak
\noindent
Then the left child of $v_{root}$ is a Wavelet Tree for $S_0[1,n_0]$ over alphabet $[a .. \lfloor (a + b)/2 \rfloor]$ and right child of $v_{root}$ is a Wavelet Tree for $S_1[1,n_1]$ over alphabet $[1 + \lfloor (a + b)/2 \rfloor .. b]$.
\end{mdframed}






