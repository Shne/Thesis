\section{Related Work}
The wavelet tree was first introduced in 2003 by Grossi, Gupta, and Vitter~\citeA[Section 4.2]{Grossi:2003:HET:644108.644250} as a way to obtain faster rank and select query times on compressed suffix arrays while maintaining empirical entropy compression.

Gonzalo Navarro~\citeA{Navarro:2014:WT:2592317.2592708} explains how the wavelet tree has many and wide ranging useful applications, from string processing including compression, full-text indexes and inverted indexes to geometry processing including various queries and computations on point grids and rectangle sets as well as graphs. 
In \citeA[Section 9]{Navarro:2014:WT:2592317.2592708} it is also mentioned that there are other data structures that achieves better time complexity than the wavelet tree, but the wavelet tree is more practical and easy to understand and implement.

Cristos Makris~\citeA{WTSurvey} also describes several effective uses for a wavelet tree, including viewing it as a range searching data structure for e.g. minimum bounding volumes and effective storage compression.
Using the wavelet tree as a compressing data structure is mainly about using various ways of encoding the bitmaps, such as using run-length encoding (RLE) on the bitmaps and storing the Burrows-Wheeler transformation (BWT) of the input string, or using Huffman Coding to shape the tree.
The Burrows-Wheeler transformation was introduced by Burrows and Wheeler~\citeA[Abstract]{BWToriginalArticle} in 1994.
Ferragina et al.~\citeA[Section~2]{waveletTreeEntropy} describes in more detail how BWT can be used to reduce the problem of compressing higher-order entropy to a problem of compressing 0-order entropy, which the wavelet tree then can do using RLE.
Mäkinen and Navarro~\citeA[Section~4]{FMcountOnBWT} invented the Huffman-shaped wavelet tree and describes in short the general principle of it without going into much detail.

Another use of a wavelet tree is answering Range Quantile queries and is described by Gagie et al.~\citeA[Section 3]{RangeQuantileQueries}.

Claude and Navarro~\citeA[Section~2.2]{Claude08practicalrankselect} give a good description of how rank and select queries is performed on the wavelet tree in practice.

Julian Shun~\citeA{DBLP:journals/corr/Shun14} describes various parallelized algorithms for constructing the wavelet tree by utilizing the GPU, achieving up to a 27x speedup over the sequential construction algorithm.

Alex Bowe~\citeA{MultiaryWaveletTreesInPractice} describes how a multiary wavelet tree can be combined with an RRR structure to support faster queries than a binary wavelet tree can accomplish using an RRR structure invented by Raman et. al. \citeB{DBLP:journals/corr/abs-0705-0552}.
An RRR structure allows computation of binary rank in $O(1)$ time and provides zero-order entropy compression for binary strings.
By using a multiary wavelet tree with the RRR structure it is possible to achieve higher order entropy compression.




