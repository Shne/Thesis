\section{Related Work}
In our implementation of rank and select we have improved binary rank and binary select using $popcount$.
\cite{Gonzalez05practicalimplementation} has also looked at improving binary rank and binary select using $popcount$. 
In section 1.1 they describe a constant time binary rank algorithm that requires a pre-computed look-up table to compute the rank. 
This table uses quite a lot of memory space. 
They improve the algorithm by decreasing the space of the lookup table by calculating it using $popcount$ which they describe in section 1.2 of their article. 

They do binary select using the same structure as rank uses: First binary search for the proper superblock using $R_s$ , then binary search that superblock for the proper block using $R_b$, and finally binary search for the position inside the block. 
This takes $O(\log n)$ time and requires little space.

\subsection{Representations of the Wavelet Tree}
The Wavelet Tree has a lot of applications that regard it in various ways.
These can be split into three main groups: A sequence of values, a reordering and a grid of points.

Representing the Wavelet Tree as a sequence of values is the most basic way to regard it. 
The Wavelet Tree simply represents the values of the sequence and supports access, rank and select queries.

The Wavelet Tree can also describe a stable reordering of the symbols in string S. 
If the leaves are traversed then all the occurrences of the smaller symbols are found first. 
Tracking a position downwards in the Wavelet Tree returns where it goes after sorting and tracking a symbol upwards tells where it is in the original string. 

A Wavelet Tree can also represent a $n \times n$ grid of \textit{n} points where no two points share the same row or column. 
One can map a general set of n points to such a discrete grid and then store the real points somewhere else.

If we have to points sorted by the x-coordinate then $S[1,n] = y_1,y_2,...,y_n$.
We can then find the x-coordinate of the i-th point by accessing $S[i]$ and the point can be found in y-coordinate order by tracking upwards from the i-th point in the leaves because the wavelet tree is representing a reordering of the points with relation to the y-coordinate.

\subsection{Applications}
