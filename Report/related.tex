\section{Related Work}
The wavelet tree was first introduced in 2003 by Grossi, Gupta, and Vitter~\citeA[Section 4.2]{Grossi:2003:HET:644108.644250} as a way to obtain faster rank and select query times on compressed suffix arrays while maintaining empirical entropy compression.

Later, and Gonzalo Navarro~\citeA{Navjda13} explains how the wavelet tree has many and wide ranging useful applications, from string processing including full-text indexes and inverted indexes to geometry processing including various queries and computations on point grids and rectangle sets as well as graphs.

Cristos Makris~\citeA{WTSurvey} also describes several effective uses for a wavelet tree, including viewing it as a range searching data structure for e.g. minimum bounding volumes and effective storage compression.
Using the wavelet tree as a compressing data structure is mainly about using various ways of encoding the bitmaps, such as using run-length encoding (RLE) on the bitmaps and storing the Burrows-Wheeler transformation (BWT) of the input string, or using Huffman Coding to shape the tree.
The Burrows-Wheeler transformation was introduced by Burrows and Wheeler~\citeA[Abstract]{BWToriginalArticle} in 1994.
Ferragina et al.~\citeA[Section~2]{waveletTreeEntropy} describes in more detail how BWT can be used to reduce the problem of compressing higher-order entropy to a problem of compressing 0-order entropy, which the wavelet tree then can do using RLE.
Mäkinen and Navarro~\citeA[Section~4]{FMcountOnBWT} invented the Huffman-shaped wavelet tree and describes in short the general principle of it without going into much detail.

Another use of a wavelet tree is answering Range Quantile queries and is described by Gagie et al.~\citeA[Section 3]{RangeQuantileQueries}.



Claude and Navarro~\citeA[Section~2.2]{Claude08practicalrankselect} give a good description of how rank and select queries is performed on the wavelet tree in practice.