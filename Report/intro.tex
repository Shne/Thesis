\section{Introduction}
The Wavelet Tree is a relatively new, but versatile data structure, offering solutions for many problem domains such as string processing, computational geometry, and data compression.
Storing, in its basic form, a sequence of characters from an alphabet it enables higher-order entropy compression and supports various fast queries.

In this thesis we have made a short survey of some of the various applications of a wavelet tree including uses in compression and in information retrieval.
We include descriptions of how the construction of a wavelet tree and its supported queries work in practice.

The practical implementation of a wavelet tree is susceptible, like all other algorithms, to the characteristics and imperfections of modern computer architectures that can degrade the performance by various penalties.
We describe and analyse how and why these characteristics give rise to these penalties.

We have implemented and tested the construction of a wavelet tree, comparing it to the theoretical running time.
We also implemented and tested the rank and select queries and performed a number of attempts at optimizing their running times by changing how they are calculated, changing the shape of the tree, changing what is stored and how it is stored.
We test and compare these optimizations including analysing how they perform with regards to the various penalties found in modern CPUs.

We first implemented the basic construction algorithm based on the description by Navarro~\citeA[Section 2]{Navarro:2014:WT:2592317.2592708}, then expanded the implementation in various ways to attempt to improve the query algorithms.

Our focus has been to implement something that could be useful in real world scenarios and we have used inputs we believe correspond to realistic use cases.
We have also avoided impractical optimizations such as ones that require recompilation to handle different sizes of alphabets.

The Wavelet Tree is a tree structure of bitmaps.
It was invented by Grossi, Grupta and Vitter~\citeA{Grossi:2003:HET:644108.644250} in 2003.
In its basic form, it is a balanced binary tree of bitmaps, encoding a \textit{sequence} or \textit{string} $S[1,n] = c_1c_2c_3 \ldots c_n$ of \textit{symbols} or \textit{characters} $c_i \in \Sigma$, where $\Sigma = [1 \ldots \sigma]$ is the \textit{alphabet} of $S$, in such a way that it supports a number of fast queries on $S$.
A balanced wavelet tree over a string $S$ with alphabet $\Sigma$ will have height $h = \lceil \log \sigma \rceil$, and $2 \sigma - 1$ nodes, with $\sigma$ of those as leaf nodes and $\sigma - 1$ as internal nodes.
In this thesis, when we write $\log$ we actually mean $\log_2$ unless otherwise noted.

The wavelet tree supports access, rank and select queries.
An access($p$) query on a wavelet tree construced on string $S$ is the query for what character $c$ is at position $p$ in the string $S$.
The rank of a character $c$ in a string $S$ up to position $p$ is written as rank$_{p}(c)$ and is defined as the number of occurrences $o$ of $c$ in the substring $S[0, \ldots, p]$.
The position of the $o$th occurrence of a character $c$ can be found with a select$_c(o)$ query.

With extensions, a wavelet tree can be used for efficient compression of $S$ while still supporting the same queries, although not as fast.
It has applications in many areas, from string processing to geometry, and can be used to represent, among others, a sequence of elements, a reordering of elements or a grid of points \citeA[Section~4]{Navarro:2014:WT:2592317.2592708}. 
When Grossi et al.~\citeA{Grossi:2003:HET:644108.644250} invented the Wavelet Tree, it was a milestone in compressed full-text indexing even though it is mentioned little in the paper.
The wavelet tree has even been shown to be able to get close to a lower bound of compression called $k$th-order entropy encoding, and we discuss this in Section~\ref{sec:entropy}.