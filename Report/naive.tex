\section{Naïve Algorithm}
This is the naive implementation we did before any smart ideas and optimizations.
\subsection{Algorithm Description}
The Naïve Wavelet Tree construction algorithm is recursively defined, calling itself to construct the left and right subtree from the root node and down. At each recursion the algorithm splits the given alphabet in two halves and traverses the given string putting each character into a left or right partition based on whether the character was in the left or right half of the alphabet.

\noindent\rule{\textwidth}{0.5pt}
\begin{algorithmic}
\Function {ConstructNode} {$String, Alphabet$}
\If{$Alphabet.Size() = 1$ or $String.Length() = 0$}
	\State $Return$
\EndIf
\State Split $Alphabet$ into $LeftAlphabet$ and $RightAlphabet$
\State $Split \gets$ middle character in $Alphabet$
\ForAll {$Character$ in $String$}
	\If {$Character < Split$}
		\State $LeftString.Append(Character)$
		\State $Bitmap.Append(0)$
	\Else
		\State $RightString.Append(Character)$
		\State $Bitmap.Append(1)$
	\EndIf
\EndFor
\State $LeftNode \gets$ \Call {ConstructNode} {$LeftAlphabet, LeftString$}
\State $RightNode \gets$ \Call {ConstructNode} {$RightAlphabet, RightString$}
\EndFunction

\State \Call {ConstructNode} {InputString, InputAlphabet}
\end{algorithmic}
\noindent\rule{\textwidth}{0.5pt}
\linebreak

\noindent In our implementation, $Alphabet$, $LeftAlphabet$, and $RightAlphabet$ are stored as two integer values each: a minimum and a maximum. It is explained in~\ref{sec:UsingIntAsChar} how this is equivalent to storing the full alphabet and passing pointers into it around. $Bitmap$ is stored as a \texttt{vector<bool>} which is tightly packed, only using 1 bit per bool\footnote{\url{http://www.cplusplus.com/reference/vector/vector-bool/}}.

\subsection{Experiments}