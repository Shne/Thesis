\section{Notes on Implementation}

\subsection{Using Integers as Characters}
\label{sec:UsingIntAsChar}
The Wavelet Tree is a data structure for strings. 
Using the C++ \texttt{char array} or C++11 \texttt{string} types would seem natural in this case, but they each have problems.
The C and C++ \texttt{char} type is only of size 1 byte allowing us only to use an alphabet size of up to 256.
This makes testing the dependency of the running times on alphabet size difficult, as we believe inaccuracies in the running time would exceed the difference in running time between the available sizes of the alphabet.

The C++11 \texttt{string} and arrays of type \texttt{char32\_t} does not have this problem and supports character types up to 32-bit unsigned. 
The problem then lies in output and readability as characters corresponding to byte values below 32 are special non-printable control characters such as carriage-return and backspace. 
At higher byte values other non-printable control characters and otherwise unreadable characters appear again. This means we would have to be selective with the allowed byte values in our alphabets if we want it to be readable for output and debugging, thereby ending up with an alphabet that is non-continuous on the set of byte values as a result, which is inconvenient.
Because of this, we have for convenience chosen to simply use vectors of integers as our strings in our implementations.
E.g. we will use \texttt{uint} instead of \texttt{char32\_t}, which both take 4 bytes of memory.
We believe this will have no impact on performance as both characters and integers are simply different representations of byte values.

In our implementation, we assume that the alphabet is always continuous on the sorted set of byte values, i.e. the alphabet spans all values possible between some minimum and maximum value, with no gaps.
Thus, we store the alphabet as a minimum and maximum value, instead of storing each value in some data structure to pass around or point into.
This is for convenience as any other non-continuous alphabet could simply be mapped to a continuous run of byte values and used in the same way. 
This mapping could e.g. be done by storing an array of the alphabet in sorted order and using pointers into this array to signify the characters. 
Lookup into the array is not necessary unless printing for human reading, since comparison of the pointer addresses returns the same result as comparing the bytes.

We will still use the terms “character/symbol” and “string” in our descriptions of the algorithms even though we have implemented them as integers and integer arrays, as we feel the terms “character/symbol” and “string” are more intuitive and give clarity.


\subsection{Generating the Data}
We implemented a small script in Python to generate our input strings of 4-byte integer values and write them in binary format to files.
This was slower than e.g. piping from \texttt{/dev/random} into a file, but we needed to constrain the alphabet and even though slow, a script was the easiest way to achieve that.

\subsection{Reading Input}
At first the input data was read from stdin using the \texttt{getline(cin, \&string)} function. 
Once we applied a profiler we found this to be horrendously slow, our Naïve algorithm spending about 20\,\% of its running time on resizing IO buffers. 
We then switched to using the \texttt{ifstream} class and IO time was reduced significantly to below 1\,\% of total running time.

\subsection{Verifying the Results}
To ensure that our implementations are correct, we implemented some simple and slow algorithms in python to calculate rank and select on the same input data we construct the wavelet trees on.
The point being that the python implementation should be so simple and easy to understand that it cannot contain errors and therefore produce the correct results for comparison.
We then compare results from rank and select queries on our wavelet tree to results from the same queries using the python implementation.
When they agree on several randomly selected sets of query parameters, we feel confident that our wavelet tree construction, rank, and select implementations are correct.

\subsection{Combating Over-Optimization}
The C++ compiler (g++) in the GNU Compiler Collection (GCC) is an optimizing compiler and can sometimes using static analysis recognize that the results and possible side-effects of a computation will not be used in the code and will in those cases completely remove that computation from the compiled code as an optimization.
This means that the compiler could potentially remove the parts of or the entire computation for our queries when we test them, if the results are not used for anything.
To ensure that the compiler does not throw needed computations out the window in our tests, the results of each query is collected in an array and printed to stdout. It is only printed after the collection of measurements is done to affect the running time minimally.

\subsection{Reducing Construction Time Memory Usage}
Since the Wavelet tree is a recursively defined data structure, we also implement it recursively.
This causes any stack-allocated variables to be held in memory until we leave the scope of the constructor function.
We traverse and split the input string into its left and right parts in each node constructor and thus end up holding the input string twice in memory: once in the variable holding the input string and once in the two variables holding the left and right split strings.
This is wasted memory because the input string is not actually needed any longer once we have split it into its left and right parts.
Because one sub-node constructor is simply called first and then the other when the first has completed and finally return once both subnodes has completed constructing themselves, we end up completing the construction of the nodes in post-order.
This means the scopes of the root node and those near the root is kept alive for most of the running time of the construction algorithm, and much memory is wasted.
The solution is to allocate these strings on the heap instead, passing pointers to the subnode constructors and having them delete them (as their input strings) once they have split them.
Doing this reduced the memory usage so much that we could run it for input strings with a length above $10^8$ characters without exhausting the 8GB available memory on our test machine.


\subsection{Bitmap implementation choice}
There are several bitmap implementations available to us. In the Standard Templating Libary (STL) of C++ there is \texttt{std::bitset<size\_t N>} and \texttt{std::vector<bool>}. From the Boost library there is \texttt{boost::dynamic\_bitset<>}.
\begin{description}
\item[\texttt{std::bitset}] While it would technically be possible to use the \texttt{std::bitset}, it requires that the size of the bitset is known at compile time and passed as a template parameter. This means it would be necessary to recompile the program for each size $n$ of the input string. 
It would also be necessary to allocate a bitmap with room for $n$ bits for each sub-node as that is the theoretically possible size required, making the size required for the bitmaps of the tree $O(n \cdot |nodes|) = O(n \cdot \sigma)$ instead of $O(n \cdot h) = O(n\log(\sigma))$.
Another reason why we cannot use \texttt{std::bitset} is because it does not support pointer access, which means that it is impossible to do queries using \texttt{popcount}, which is a CPU instruction we utilize to improve the practical running time of \textproc{Rank} and \textproc{Select} queries and is described in Section~\ref{sec:simpleoptimizations}.
We also believe that an actual usable practical implementation should be able to handle different sizes of input at run time instead of compile time. 

\item[\texttt{std::vector<bool>}] is a specialised implementation for \texttt{bool} that packs the data so that each \texttt{bool} only takes up one bit and is not an actual C++ container, though it tries to mimic some of the behaviour. 
It is basically the STL implementation of a dynamically allocated bitset. This is the implementation we decided to use for our bitmap because it allows dynamic allocation and pointer access.

\item[\texttt{boost::dynamic\_bitset}] is the Boost library's take on a dynamic bitset. 
It does not try to mimic a container and lacks some features such as an iterator because of that. 
It also does not guarantee that the bits will be allocated consecutively in memory and has no raw pointer access to the data in memory. 
This is a problem when calling popcount on all machine words from beginning up to some index.
\end{description}

We chose to use \texttt{vector<bool>} mainly because it supported direct pointer access into the backing array and that backing array was a single continuous array so we could do pointer arithmetic across an entire bitmap.
\texttt{boost::dynamic\_bitset} does not support any of this.

Pieterse et al.~\citeB{Pieterse:2010:PCB:1899503.1899530} has tested 5 different bitvector implementations for C++, including \texttt{std::vector<bool>} and \texttt{boost::dynamic\_bitset}.
In terms of running time, \texttt{std::vector<bool>} performs the worst of the tested bitvectors for their test case.
But, their test case does not at all resemble the way we use it.
We do not utilize any of the extra functions or features they have tested such as the reset operation and bitwise operations on entire arrays.

In terms of memory, \texttt{std::vector<bool>} performed the best by using the least amount of memory, owing to its simple implementation storing no additional meta information and using only a single raw array as its backing data format.
This is a characteristic we like for our purposes as we basically use it as an array with easier access to single bit values.

\subsection{Challenges in Implementation}
The wavelet tree is a somewhat simple data structure as a tree structure of bitmaps implemented using pointers and dynamic bitsets.
The construction of the wavelet tree was not a great challenge, neither was the basic forms of rank and select queries.
But, in later iterations of our wavelet tree, we implemented more and more intricate designs of the bitmaps, spending more and more time debugging to make it work absolutely correctly.

We are not experts in C++ and have in fact programmed very little in it previously, which both introduced and complicated many issues that someone with more experience with C++ would likely have had little trouble with.
The sheer number of different algorithms we implemented for the various variations of the wavelet tree only exacerbated the time spent implementing and debugging.

Notable challenges include implementing rank and select queries that utilized concatenated bitmaps, implementing support for aligning the precomputed blocks with machine pages, and handling and masking machine word correctly when using the cpu intrinsic popcount instruction.


\begin{description}
\item[Popcount Instruction] works on whole machine words at a time and so our code had to figure out when it was worth using the instruction and handle any excess bits counted or any bits not counted.
\item[Concatenated Bitmaps] required that our code correctly handled the edge cases where bitmaps touch each other, making sure to only count the number of 0s or 1s on the correct side of the boundary.
This was done by using pointer arithmetic and calculating various offsets, misalignments and offsets of offsets and misalignments.
It only became more complicated when using precomputed values as well.
\item[Page-aligned Blocks] only introduced more handling and bookkeeping of offsets and offsets of offsets.
\end{description}
