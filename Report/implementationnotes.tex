\section{Notes on Implementation}

\subsection{Using Integers as Characters}
\label{sec:UsingIntAsChar}
The Wavelet Tree is a data structure for strings. 
Using the C++ \texttt{char array} or C++11 \texttt{string} types would seem natural in this case, but they each have problems.
The C and C++ \texttt{char} type is only of size 1 byte allowing us only to use an alphabet size of up to 256, making testing the running times dependency on alphabet size difficult as we believe inaccuracies in the running time would exceed the difference in running time between the available sizes of the alphabet.

The C++11 \texttt{string} and arrays of type \texttt{char32\_t} does not have this problem and supports character types up to 32-bit unsigned. 
The problem then lies in output and readability as characters corresponding to byte values below 32 are special non-printable control characters such as carriage-return and backspace. 
At higher byte values other non-printable control characters and otherwise unreadable characters appear again, meaning we would have to be selective with the allowed byte values in our alphabets if we want it to be readable for output and debugging, and end up with an alphabet that is non-continuous on the set of byte values as a result.
Because of this, we have for convenience chosen to simply use arrays of integers as our strings in our implementations.
This will have no impact on performance as both characters and integers are simply different representations of byte values, so.

We assume in our implementation that the alphabet is always continuous on the set of byte values and store the alphabet as a minimum and maximum value, instead of storing each value in some data structure to pass around or point into.
This is for convenience as any other non-continuous alphabet could simply be mapped to a continuous run of byte values and used in the same way. 
This mapping could e.g. be done by storing an array of the alphabet in sorted order and using pointers into this array to signify the characters. 
Lookup into the array is not necessary unless printing for human reading, since comparison of the pointer addresses returns the same result as comparing the bytes.

We will still use the terms “character/symbol” and “string” in our descriptions of the algorithms even though we have implemented them as integers and integer arrays, as we feel the terms “character/symbol” and “string” are more intuitive and give clarity.


\subsection{Generating the Data}
We implemented a small script in Python to generate our input strings and write them in binary format to files.
This was slower than e.g. piping from \texttt{/dev/random} into a file, but we needed to constrain the alphabet and even though slow, a python script was the easiest way to achieve that.

\subsection{Reading Input}
At first the input data was read from stdin using the \texttt{getline(cin, \&string)} function. 
Once we applied a profiler we found this to be horrendously slow, our Naïve algorithm spending about 20\% of its running time on resizing IO buffers. 
We then switched to using the \texttt{ifstream} class and IO time was reduced significantly to below 1\% of total running time.

\subsection{Verifying the Results}
To ensure that our implementations are correct, we implemented some simple and slow algorithms in python to calculate rank and select on the same input data we construct the wavelet trees on.
The point being that the python implementation should be simple and easy to understand and therefore produce the correct results for comparison.
We then compare results from rank and select queries on our wavelet tree to results from the same queries using the python implementation.
When they agree on several randomly selected sets of query parameters, we feel confident that our wavelet tree construction, rank, and select implementations are correct.

\subsection{Combating Over-Optimization}
The GNU Compiler Collection (GCC) is an optimizing compiler and can sometimes using static analysis recognize that the results and possible side-effects of a computation will not be used in the code and will in those cases completely remove that computation from the compiled code as an optimization.
This means that the compiler could potentially remove the parts of or the entire computation for our queries when we test them, if the results are not used for anything.
To ensure that the compiler does not throw needed computations out the window in our tests, the results of each query is collected in an array and printed to stdout. It is only printed after the collection of measurements is done to effect the running time minimally.

\subsection{Reducing Construction Time Memory Usage}
Since the Wavelet tree is a recursively defined data structure, we also implement it recursively.
This causes any stack-allocated variables to be held in memory until we leave the scope of the constructor function.
We traverse and split the input string into its left and right parts in each node constructor and thus end up holding the input string twice in memory: once in the variable holding the input string and once in the two variables holding the left and right split strings.
This is wasted memory because the input string is not actually needed any longer once we have split it into its left and right parts.
Because one sub-node constructor is simply called first and then the other when the first has completed and finally return once both subnodes has completed constructing themselves, we end up completing the construction of the nodes in post-order.
This means the scopes of the root node and those near the root is kept alive for most of the running time of the construction algorithm, and much memory is wasted.
The solution is to allocate these strings on the heap instead, passing pointers to the subnode constructors and having them delete them (as their input strings) once they have split them.
Doing this reduced the memory usage so much that we could run it for input strings with a length above $10^8$ without exhausting the 8GB available memory on our test machine.


\subsection{Bitmap implementation choice}
There are several bitmap implementations available to us. In the Standard Templating Libary (STL) of C++ there is \texttt{std::bitset<size\_t N>} and \texttt{std::vector<bool>}. From the Boost library there is \texttt{boost::dynamic\_bitset<>}.
\begin{description}
\item[\texttt{std::bitset}] While it would technically be possible to use the \texttt{std::bitset}, it requires that the size of the bitset is known at compile time and passed as a template parameter. This means it would be necessary to recompile the program for each $n$, or size of the input string. 
It would also be necessary to allocate a bitmap with room for $n$ bits for each sub-node as that is the theoretically possible size required, making the size required for the bitmaps of the tree $O(n \times |nodes|) = O(n2^{\log(\sigma)})$ instead of $O(n \times height) = O(n~\log(\sigma))$.
Another reason why we cannot use \texttt{std::bitset} is because it does not support pointer access, which means that it is impossible to do queries using \texttt{popcount}, which is a CPU instruction we utilize to improve the practical running time of \textproc{Rank} and \textproc{Select} queries and is described in Section~\ref{sec:simpleoptimizations}.
We also believe that an actual usable practical implementation should be able to handle different sizes of input at runtime instead of compile-time. 

\item[\texttt{boost::dynamic\_bitset}] is the Boost library's take on a dynamic bitset. 
It does not try to mimic a container and lacks some features such as an iterator because of that. 
It also does not guarantee that the bits will be allocated consecutively in memory and has no raw pointer access to the data in memory. 
This is a problem when calling popcount on all machine words from beginning up to some index.

\item[\texttt{vector<bool>}] is a specialised implementation for \texttt{bool} that packs the data so that each \texttt{bool} only takes up one bit and is not an actual C++ container, though it tries to mimic some of the behaviour. 
It is basically the STL implementation of a dynamically allocated bitset. This is the implementation we decided to use for our bitmap because it allows dynamic allocation and pointer access.
\end{description}

