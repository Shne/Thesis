\section{Notes on The Experiments}
Here we discuss some things general for all our experiments, or all those where applicable.

\subsection{Choice of Input String}
We have chosen to construct the input strings used in our experiments so that each character occurs with the same probability at each position.
This means the string has a uniform distribution of characters from the alphabet.
We have chosen to do so for several reasons, among them being that we believe it to be a realistic use case, as well as making the choice of character to query for in our experiments make less difference.
The even amount of occurrences of each character also means there will be little difference in size of bitmap among the nodes in a single layer of the tree.


\subsection{Choice of Query Parameters}
It is important to ensure that we don't introduce a bias in our experiments on the rank and select query performances by our choice of query parameters.
As we have chosen to use a randomly generated input string with uniform distribution of characters, there should be little difference in the frequency of characters and little difference in query performance based on the exact choice of character.
There is, however a difference of where in the tree the node each character corresponds to, and we should make sure to use characters from various positions in the alphabet, to have the queries together traverse as much of the tree as possible.

For the rank queries there is also the position parameter, determining how far into the string the query should look and therefore how far into each bitmap the query should look.
A high value (close to the length of the string) might seem like a good idea to make the query go through most of the bitmaps, but we don't want to introduce a bias by using some constant high value, nor do we want to risk introducing a bias by only looking at high values for the position parameter.
Again we choose to use values from all parts of the range of valid values for the parameter.

We are also interested in avoiding introducing any bias by using only one type of combination of parameters.
If we had e.g. let both parameter values depend on the index of a single for-loop around the call to the query, we would have only tested low character values together with low position values as well as high character values together with high position values.

Instead we let one parameter ascend from valid low values to valid high values with even spacing to reach the highest valid value in the lastly performed query. Meanwhile, the other parameter increases more rapidly with wider spacing, and then wraps around before passing highest valid value to then start again at low values, with an offset to not repeat parameter values, doing so many times before the end.
This ensures we perform the queries for all combinations of high, medium and low parameter values in our experiments.
