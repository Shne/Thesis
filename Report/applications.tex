\section{Applications}
\subsection{What The Wavelet Tree Can Represent}
\textbf{[TODO: redo this subsection]}
The Wavelet Tree has multiple applications that each utilize the wavelet tree differently and use it for storage of, and queries on, different types of data.
These can be split into three main types: A sequence of values, a reordering or permutation, and a grid of points.

Using the Wavelet Tree to store a sequence of values is perhaps the most basic way to utilize the tree.
The Wavelet Tree stores the values and their sequence and supports access, rank, and select queries.

The Wavelet Tree can also be used to describe a stable reordering, e.g. of the symbols in a string S. 
If the leaves are traversed then all the occurrences of the smaller symbols are found first. \textbf{[TODO: describe the use of traversing the leaves like so]}
By tracking the position of a symbol downwards through the Wavelet Tree we will find the new position of the symbol after the permutation.
Tracking the position of a symbol upwards through the wavelet tree will where it is in the original string. 

A Wavelet Tree can also represent a $n \times n$ grid of \textit{n} points where no two points share the same row or column. 
One can map a general set of n points to such a discrete grid and then store the real points somewhere else.

If we have points sorted by the x-coordinate and take only the y-coordinates $S_y[1,n] = y_1,y_2,...,y_n$ and save $S_y$ in a Wavelet Tree we can the find the \textit{i}th point in x-coordinate order by accessing the corresponding y-coordinate in the wavelet tree. 
If we want the \textit{i}th point i y-coordinate order we can access the leaf of a given y-coordinate and find it's corresponding x-coordinate by querying up through the tree until we find the original position of y in S. 
The corresponding x-coordinate will be at the same position.
Querying from a leaf gives the points i y-coordinate order because the leaves are sorted by y-coordinate.

\subsection{Information Retrieval}
\textbf{[TODO: overview of Information Retrievel Applications of Wavelet Trees]}

\subsection{Compression}
\textbf{[TODO: overview of Compression Applications of Wavelet Trees]}
