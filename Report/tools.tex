\section{Tools Used}
We have looked at and tested out the capabilities of several profilers and tools for determining the number of cache-misses, branch mispredictions and translation lookaside buffer misses.
Below we will explain why we did or did not use each of them.


\subsection{Counters}
We use these for reporting and analysing cache-misses, etc.
\begin{description}
\item[Perf]~\citep{perftool} is a performance analysing tools chiefly implemented in the Linux kernel, available from version 2.6.31.
It supports reading and reporting various counters from the \textit{hardware}, meaning it doesn't emulate the CPU or similar, as some other tools do.
It can profile the entire system or a specific process, but not subsections of a program.
\item[PAPI]~\citep{PAPI} (Performance Application Programming Interface) can use the perf kernel driver but pre-dates perf.
It requires the analysed program itself to setup and initialize PAPI, but therefore also supports starting and stopping the counter data collection at specific points in the program, enabling profiling of subsections of the program.
\end{description}

\subsection{Profiler}
We use these for finding hotspots in our code; which parts our program is spending most of its time in and therefore which parts we should try to optimize.

\begin{description}
\item[Zoom]~\citep{zoomprofiler} is a profiler that supports profiling any running program without modifcation, showing a callgraph with time spent in each call among other things.
It uses the perf kernel driver. It was fairly easy to use, but seemed somewhat limited in its capabilities.
Having to start Zoom profiling before running the program, stopping Zoom when program was done and then finding the program in a long list instead of wrapping the program in the Zoom profiler somehow was also a hassle.
\item[Callgrind]~\citep{callgrind} is a callgraph analyser tool in the Valgrind suite.
It supports wrapping a program, so no manual intervention is needed.
Valgrind compiles the analysed program into an intermediate representation and runs that completely in a virtual machine to extract information for its tools.
This causes the program to run much slower while being analysed, but this is a minor concern for us.
Callgrind outputs a .callgrind file which can then be viewed in the kCacheGrind GUI program.
\end{description}